\documentclass[12pt]{article}

% xelatex -shell-escape -interaction=nonstopmode main.tex

% 无 twoside,则生成用于手机或电脑阅读的版本。
% 有 twoside,则生成用于 A4 双面打印的版本。然后再用以下 pdfjam 命令行处理,接着用 MacOS 预览 out.pdf,把偶数页选中,然后都旋转180度。
% pdfjam --nup 2x1 --landscape main.pdf '4,1-3,8,5-7,12,9-11,16,13-15,20,17-19,24,21-23,28,25-27,32,29-31,36,33-35,40,37-39,44,41-43,48,45-47,52,49-51,56,53-55,60,57-59,64,61-63,68,65-67,72,69-71,76,73-75,80,77-79,84,81-83,88,85-87,92,89-91,96,93-95,100,97-99,104,101-103'  --outfile out.pdf

\usepackage{ctex}
\usepackage[a6paper,twoside]{geometry}
%\usepackage[a6paper]{geometry}
\usepackage{fontspec}
\usepackage{titletoc}
\usepackage{marginnote}
\usepackage{xpinyin}

% 设置字体
\setmainfont{Times New Roman}
\setCJKmainfont{Heiti SC} % 黑体
\setCJKfamilyfont{kai}{Kaiti SC} % 楷体
\setCJKfamilyfont{fangsong}{STFangsong}  % 仿宋体

\newfontfamily\PinyinFont{印氪先生汉语拼音 人教版W3}

% 自定义标签字体
\newcommand \poet{\CJKfamily{fangsong}\zihao{-4}}
\newcommand \poetry {\CJKfamily{kai}\zihao{4}}
\newcommand \vs {\vspace{2pt}}

% 设置旁注的字体为10pt的楷体,并且与正文的距离为1cm
\renewcommand*{\marginfont}{\normalsize}
\setlength{\marginparsep}{0cm}
\begin{document}

\begin{titlepage}
    \centering
    \vspace*{1cm}
    {\huge 小学生必背古诗词}

    \vspace{1cm}
    {\large \mbox{部编版课本114首,增补56首}}

    \vfill

    %{\large \today}
    {\large 2024年3月29日}

    \date{}
\end{titlepage}

% 空白页,不编页码
\thispagestyle{empty}
\mbox{}

% 第三页开始编页码
\clearpage
\setcounter{page}{1}
\pagestyle{plain}


\titlecontents{section}
  [0em] % 左边距

\titlecontents{subsection}
  [1.6em] % 左边距
  {\vspace{0em}} % 编号与标题之间的间距
  {\thecontentslabel\hspace{1em}} % 编号的格式
  {} % 无编号的格式
  {\titlerule*[1pc]{.}\contentspage} % 点号和页码

{\footnotesize \tableofcontents}

\newpage

\centering

\xpinyinsetup{ratio={.7},font={\PinyinFont}}

\section{一年级上册}

\subsection{咏鹅}

\poet[唐]骆宾王 \vs

\poetry

鹅,鹅,鹅,

\marginnote{\xpinyin{曲}{qu1}}

曲项向天歌。

白毛浮绿水,

红掌拨清波。

\subsection{江南}

\poet 汉乐府 \vs

\poetry

江南可采莲,莲叶何田田。

\marginnote{\xpinyin{间}{jian1}}

鱼戏莲叶间。

鱼戏莲叶东,鱼戏莲叶西,

鱼戏莲叶南,鱼戏莲叶北。

\subsection{画}

\poet[唐]王维 \vs

\poetry

远看山有色,近听水无声。

\marginnote{\xpinyin{还}{hai2}}

春去花还在,人来鸟不惊。

\subsection{悯农(其二)}

\marginnote{\xpinyin{悯}{min3} \xpinyin{绅}{shen1}}

\poet[唐]李绅 \vs

\poetry

\marginnote{\xpinyin{当}{dang1}}

锄禾日当午,汗滴禾下土。

\marginnote{\xpinyin{谁}{shui2}}

谁知盘中餐,粒粒皆辛苦。

\subsection{古朗月行(节选)}

\poet[唐]李白 \vs

\poetry

小时不识月,

呼作白玉盘。

\marginnote{\xpinyin{瑶}{yao2}}

又疑瑶台镜,

飞在青云端。

\subsection{风}

\marginnote{\xpinyin{峤}{qiao2}}

\poet[唐]李峤 \vs

\poetry

解落三秋叶,

能开二月花。

过江千尺浪,

入竹万竿斜。

\newpage

\section{一年级下册}

\subsection{春晓}

\marginnote{\xpinyin{孟}{meng4}}

\poet[唐] 孟浩然 \vs

\poetry

春眠不觉晓,

\marginnote{\xpinyin{啼}{ti2}}

处处闻啼鸟。

夜来风雨声,

花落知多少。

\subsection{赠汪伦}

\poet[唐]李白 \vs

\poetry

李白乘舟将欲行,

忽闻岸上踏歌声。

桃花潭水深千尺,

不及汪伦送我情。

\subsection{静夜思}

\poet[唐]李白 \vs

\poetry

床前明月光,

疑是地上霜。

举头望明月,

低头思故乡。

\subsection{寻隐者不遇}

\marginnote{\xpinyin{隐}{yin3} \xpinyin{贾}{jia3}}

\poet[唐]贾岛  \vs

\poetry

松下问童子,

言师采药去。

只在此山中,

云深不知处。

\subsection{池上}

\poet[唐]白居易 \vs

\poetry

\marginnote{\xpinyin{娃}{wa2}}

小娃撑小艇,

偷采白莲回。

\marginnote{\xpinyin{藏}{cang2} \xpinyin{踪}{zong1}}

不解藏踪迹,

浮萍一道开。

\subsection{小池}

\poet[宋]杨万里 \vs

\poetry

泉眼无声惜细流,

树阴照水爱晴柔。

\marginnote{\xpinyin{露}{lu4}}

小荷才露尖尖角,

早有蜻蜓立上头。

\subsection{画鸡}

\poet[明]唐寅 \vs

\poetry

\marginnote{\xpinyin{寅}{yin2} \xpinyin{冠}{guan1}}

头上红冠不用裁,

满身雪白走将来。

平生不敢轻言语,

一叫千门万户开。

\newpage

\section{二年级上册}

\subsection{梅花}

\poet[宋]王安石 \vs

\poetry

墙角数枝梅,

凌寒独自开。

遥知不是雪,

\marginnote{\xpinyin{为}{wei4}}

为有暗香来。

\subsection{小儿垂钓}

\poet[唐]胡令能 \vs

\poetry

\marginnote{\xpinyin{蓬}{peng2} \xpinyin{稚}{zhi4}}

蓬头稚子学垂纶,

侧坐莓苔草映身。

路人借问遥招手,

\marginnote{\xpinyin{应}{ying4}}

怕得鱼惊不应人。

\subsection{登鹳雀楼}

\marginnote{\xpinyin{雀}{que4}}

\poet[唐]王之涣 \vs

\poetry

白日依山尽,

黄河入海流。

欲穷千里目,

更上一层楼。

\subsection{望庐山瀑布}

\poet[唐]李白 \vs

\poetry

日照香炉生紫烟,

遥看瀑布挂前川。

飞流直下三千尺,

疑是银河落九天。

\subsection{江雪}

\poet[唐]柳宗元 \vs

\poetry

千山鸟飞绝,

万径人踪灭。

\marginnote{\xpinyin{蓑}{suo1}}

孤舟蓑笠翁,

独钓寒江雪。

\subsection{夜宿山寺}

\poet[唐]李白 \vs

\poetry

危楼高百尺,

手可摘星辰。

不敢高声语,

恐惊天上人。

\subsection{敕勒歌}

\poet 北朝民歌 \vs

\poetry

\marginnote{\xpinyin{敕}{chi4} \xpinyin{勒}{le4}}

敕勒川,阴山下,

\marginnote{\xpinyin{穹}{qiong2} \xpinyin{野}{ye3}}

天似穹庐,笼盖四野。

天苍苍,野茫茫,

\marginnote{\xpinyin{见}{xian4}}

风吹草低见牛羊。

\newpage

\section{二年级下册}

\subsection{村居}

\marginnote{\xpinyin{鼎}{ding3}}

\poet[清]高鼎 \vs

\poetry

\marginnote{\xpinyin{莺}{ying1}}

草长莺飞二月天,

\marginnote{\xpinyin{拂}{fu2} \xpinyin{堤}{di1}}

拂堤杨柳醉春烟。

儿童散学归来早,

\marginnote{\xpinyin{鸢}{yuan1}}

忙趁东风放纸鸢。

\subsection{咏柳}

\poet[唐]贺知章 \vs

\poetry

\marginnote{\xpinyin{妆}{zhuang1}}

碧玉妆成一树高,

\marginnote{\xpinyin{绦}{tao1}}

万条垂下绿丝绦。

\marginnote{\xpinyin{裁}{cai2}}

不知细叶谁裁出,

\marginnote{\xpinyin{剪}{jian3}}

二月春风似剪刀。

\subsection{赋得古原草送别(节选)}

\marginnote{\xpinyin{赋}{fu4}}

\poet[唐]白居易 \vs

\poetry

离离原上草,

\marginnote{\xpinyin{荣}{rong2}}

一岁一枯荣。

野火烧不尽,

春风吹又生。

\subsection{晓出净慈寺送林子方}

\marginnote{\xpinyin{慈}{ci2}}

\poet[宋]杨万里 \vs

\poetry

\marginnote{\xpinyin{毕}{bi4} \xpinyin{竟}{jing4}}

毕竟西湖六月中,

风光不与四时同。

接天莲叶无穷碧,

\marginnote{\xpinyin{映}{ying4}}

映日荷花别样红。

\subsection{绝句}

\marginnote{\xpinyin{绝}{jue2} \xpinyin{甫}{fu3}}

\poet[唐]杜甫 \vs

\poetry

\marginnote{\xpinyin{鹂}{li2} \xpinyin{鸣}{ming2}}

两个黄鹂鸣翠柳,

\marginnote{\xpinyin{行}{hang2} \xpinyin{鹭}{lu4}}

一行白鹭上青天。

\marginnote{\xpinyin{含}{han2} \xpinyin{岭}{ling3}}

窗含西岭千秋雪,

\marginnote{\xpinyin{泊}{bo2}}

门泊东吴万里船。

\subsection{悯农(其一)}

\marginnote{\xpinyin{悯}{min3}}

\poet[唐]李绅 \vs

\poetry

\marginnote{\xpinyin{绅}{shen1} \xpinyin{粟}{su4}}

春种一粒粟,

秋收万颗子。

四海无闲田,

\marginnote{\xpinyin{犹}{you2}}

农夫犹饿死。

\subsection{舟夜书所见}

\marginnote{\xpinyin{查}{zha1} \xpinyin{慎}{shen4}}

\poet[清]查慎行 \vs

\poetry

月黑见渔灯,

孤光一点萤。

\marginnote{\xpinyin{簇}{cu4}}

微微风簇浪,

散作满河星。

\newpage

\section{三年级上册}

\subsection{所见}

\poet[清]袁枚 \vs

\poetry

牧童骑黄牛,

\marginnote{\xpinyin{樾}{yue4}}

歌声振林樾。

意欲捕鸣蝉,

忽然闭口立。

\subsection{山行}

\poet[唐]杜牧 \vs

\poetry

远上寒山石径斜,

白云生处有人家。

停车坐爱枫林晚,

霜叶红于二月花。

\subsection{赠刘景文}

\poet[宋]苏轼 \vs

\poetry

\marginnote{\xpinyin{擎}{qing2}}

荷尽已无擎雨盖,

\marginnote{\xpinyin{傲}{ao4}}

菊残犹有傲霜枝。

一年好景君须记,

\marginnote{\xpinyin{橘}{ju2}}

最是橙黄橘绿时。

\subsection{夜书所见}

\poet[宋]叶绍翁 \vs

\poetry

\marginnote{\xpinyin{萧}{xiao1}}

萧萧梧叶送寒声,

江上秋风动客情。

\marginnote{\xpinyin{挑}{tiao3} \xpinyin{促}{cu4}}

知有儿童挑促织,

\marginnote{\xpinyin{篱}{li2}}

夜深篱落一灯明。

\subsection{望天门山}

\poet[唐]李白 \vs

\poetry

天门中断楚江开,

碧水东流至此回。

两岸青山相对出,

\marginnote{\xpinyin{帆}{fan1}}

孤帆一片日边来。

\subsection{饮湖上初晴后雨}

\poet[宋]苏轼 \vs

\poetry

\marginnote{\xpinyin{轼}{shi4}}

水光潋滟晴方好,

\marginnote{\xpinyin{潋}{lian4} \xpinyin{滟}{yan4}}

山色空蒙雨亦奇。

欲把西湖比西子,

\marginnote{\xpinyin{抹}{mo3} \xpinyin{宜}{yi2}}

淡妆浓抹总相宜。

\subsection{望洞庭}

\marginnote{\xpinyin{禹}{yu3} \xpinyin{锡}{xi1}}

\poet[唐]刘禹锡 \vs

\poetry

湖光秋月两相和,

\marginnote{\xpinyin{未}{wei4} \xpinyin{磨}{mo2}}

潭面无风镜未磨。

遥望洞庭山水翠,

\marginnote{\xpinyin{盘}{pan2}}

白银盘里一青螺。

\subsection{早发白帝城}

\poet[唐]李白 \vs

\poetry

朝辞白帝彩云间,

\marginnote{\xpinyin{还}{huan2}}

千里江陵一日还。

两岸猿声啼不住,

轻舟已过万重山。

\subsection{采莲曲}

\poet[唐]王昌龄 \vs

\poetry

荷叶罗裙一色裁,

\marginnote{\xpinyin{芙}{fu2} \xpinyin{蓉}{rong2}}

芙蓉向脸两边开。

乱入池中看不见,

闻歌始觉有人来。

\newpage

\section{三年级下册}

\subsection{绝句}

\poet[唐]杜甫 \vs

\poetry

迟日江山丽,

春风花草香。

泥融飞燕子,

\marginnote{\xpinyin{鸳}{yuan1} \xpinyin{鸯}{yang1}}

沙暖睡鸳鸯。

\subsection{惠崇春江晚景}

\marginnote{\xpinyin{惠}{hui4} \xpinyin{崇}{chong2}}

\poet[宋]苏轼 \vs

\poetry

竹外桃花三两枝,

春江水暖鸭先知。

\marginnote{\xpinyin{蒌}{lou2} \xpinyin{蒿}{hao1}}

蒌蒿满地芦芽短,

\marginnote{\xpinyin{豚}{tun2}}

正是河豚欲上时。

\subsection{三衢道中}

\marginnote{\xpinyin{衢}{qu2} \xpinyin{曾}{zeng1}}

\poet[宋]曾几 \vs

\poetry

梅子黄时日日晴,

小溪泛尽却山行。

绿阴不减来时路,

添得黄鹂四五声。

\subsection{忆江南}

\poet[唐]白居易 \vs

\poetry

江南好,

\marginnote{\xpinyin{曾}{ceng2} \xpinyin{谙}{an1}}

风景旧曾谙。

日出江花红胜火,

春来江水绿如蓝。

能不忆江南?

\subsection{元日}

\poet[宋]王安石 \vs

\poetry

爆竹声中一岁除,

\marginnote{\xpinyin{屠}{tu2}}

春风送暖入屠苏。

\marginnote{\xpinyin{曈}{tong2}}

千门万户曈曈日,

总把新桃换旧符。

\subsection{清明}

\poet[唐]杜牧 \vs

\poetry

清明时节雨纷纷,

\marginnote{\xpinyin{魂}{hun2}}

路上行人欲断魂。

借问酒家何处有?

牧童遥指杏花村。

\subsection{九月九日忆山东兄弟}

\poet[唐]王维 \vs

\poetry

独在异乡为异客,

每逢佳节倍思亲。

遥知兄弟登高处,

\marginnote{\xpinyin{茱}{zhu1} \xpinyin{萸}{yu2}}

遍插茱萸少一人。

\subsection{滁州西涧}

\marginnote{\xpinyin{滁}{chu2} \xpinyin{涧}{jian4}}

\poet[唐]韦应物 \vs

\poetry

\marginnote{\xpinyin{韦}{wei2} \xpinyin{幽}{you1}}

独怜幽草涧边生,

上有黄鹂深树鸣。

春潮带雨晚来急,

野渡无人舟自横。

\subsection{大林寺桃花}

\poet[唐]白居易 \vs

\poetry

\marginnote{\xpinyin{菲}{fei1}}

人间四月芳菲尽,

山寺桃花始盛开。

\marginnote{\xpinyin{恨}{hen4} \xpinyin{觅}{mi4}}

长恨春归无觅处,

不知转入此中来。

\newpage

\section{四年级上册}

\subsection{浪淘沙(其七)}

\poet[唐]刘禹锡 \vs

\poetry

\marginnote{\xpinyin{涛}{tao1}}

八月涛声吼地来,

头高数丈触山回。

\marginnote{\xpinyin{臾}{yu2}}

须臾却入海门去,

卷起沙堆似雪堆。

\subsection{鹿柴}

\marginnote{\xpinyin{柴}{zhai4}}

\poet[唐]王维 \vs

\poetry

空山不见人,但闻人语响。

\marginnote{\xpinyin{返}{fan3} \xpinyin{苔}{tai2}}

返景入深林,复照青苔上。

\subsection{暮江吟}

\marginnote{\xpinyin{暮}{mu4}}

\poet[唐]白居易 \vs

\poetry

一道残阳铺水中,

\marginnote{\xpinyin{瑟}{se4}}

半江瑟瑟半江红。

可怜九月初三夜,

露似真珠月似弓。

\subsection{题西林壁}

\poet[宋]苏轼 \vs

\poetry

横看成岭侧成峰,

远近高低各不同。

不识庐山真面目,

\marginnote{\xpinyin{缘}{yuan2}}

只缘身在此山中。

\subsection{雪梅}

\marginnote{\xpinyin{卢}{lu2} \xpinyin{钺}{yue4}}

\poet[宋]卢钺 \vs

\poetry

\marginnote{\xpinyin{降}{xiang2}}

梅雪争春未肯降,

\marginnote{\xpinyin{骚}{sao1}}

骚人阁笔费评章。

\marginnote{同\xpinyin{搁}{ge1}}

梅须逊雪三分白,

\marginnote{\xpinyin{逊}{xun4}}

雪却输梅一段香。

\subsection{嫦娥}

\marginnote{\xpinyin{嫦}{chang2} \xpinyin{娥}{e2}}

\poet[唐]李商隐 \vs

\poetry

云母屏风烛影深,

长河渐落晓星沉。

嫦娥应悔偷灵药,

碧海青天夜夜心。

\subsection{出塞}

\marginnote{\xpinyin{塞}{sai4}}

\poet[唐]王昌龄 \vs

\poetry

\marginnote{\xpinyin{秦}{qin2}}

秦时明月汉时关,

\marginnote{\xpinyin{征}{zheng1}}

万里长征人未还。

\marginnote{\xpinyin{将}{jiang4}}

但使龙城飞将在,

不教胡马度阴山。

\subsection{凉州词}

\marginnote{\xpinyin{翰}{han4}}

\poet[唐]王翰 \vs

\poetry

葡萄美酒夜光杯,

\marginnote{\xpinyin{琵}{pi2} \xpinyin{琶}{pa2}}

欲饮琵琶马上催。

醉卧沙场君莫笑,

古来征战几人回?

\subsection{夏日绝句}

\poet[宋]李清照 \vs

\poetry

\marginnote{\xpinyin{杰}{jie2}}

生当作人杰,

死亦为鬼雄。

至今思项羽,

不肯过江东。

\subsection{别董大}

\poet[唐]高适 \vs

\poetry

\marginnote{\xpinyin{曛}{xun1}}

千里黄云白日曛,

北风吹雁雪纷纷。

莫愁前路无知己,

天下谁人不识君?

\newpage

\section{四年级下册}

\subsection{四时田园杂兴(其二十五)}

\marginnote{\xpinyin{杂}{za2} \xpinyin{兴}{xing4}}

\poet[宋]范成大 \vs

\poetry

梅子金黄杏子肥,

麦花雪白菜花稀。

日长篱落无人过,

\marginnote{\xpinyin{惟}{wei2} \xpinyin{蛱}{jia2}}

惟有蜻蜓蛱蝶飞。

\subsection{宿新市徐公店}

\marginnote{\xpinyin{徐}{xu2}}

\poet[宋]杨万里 \vs

\poetry

\marginnote{\xpinyin{疏}{shu1}}

篱落疏疏一径深,

树头新绿未成阴。

儿童急走追黄蝶,

飞入菜花无处寻。

\subsection{清平乐·村居}

\marginnote{\xpinyin{乐}{yue4} \xpinyin{媪}{ao3}}

\poet[宋]辛弃疾 \vs

\poetry

\marginnote{\xpinyin{亡}{wu2} \xpinyin{剥}{bo1}}

茅檐低小,溪上青青草。

醉里吴音相媚好,白发谁家翁媪?

大儿锄豆溪东,中儿正织鸡笼。

最喜小儿亡赖,溪头卧剥莲蓬。

\subsection{卜算子·咏梅}

\marginnote{\xpinyin{卜}{bu3}}

\poet 毛泽东 \vs

\poetry

风雨送春归,飞雪迎春到。

已是悬崖百丈冰,犹有花枝俏。

\marginnote{\xpinyin{俏}{qiao4}}

俏也不争春,只把春来报。

待到山花烂漫时,她在丛中笑。

\subsection{江畔独步寻花}

\marginnote{\xpinyin{畔}{pan4}}

\poet[唐]杜甫 \vs

\poetry

黄师塔前江水东,

\marginnote{\xpinyin{倚}{yi3}}

春光懒困倚微风。

桃花一簇开无主,

可爱深红爱浅红?

\subsection{蜂}

\poet[唐]罗隐 \vs

\poetry

\mbox{不论平地与山尖,无限风光尽被占。}

\mbox{采得百花成蜜后,为谁辛苦为谁甜?}

\subsection{独坐敬亭山}

\poet[唐]李白 \vs

\poetry

众鸟高飞尽,孤云独去闲。

相看两不厌,只有敬亭山。

\subsection{芙蓉楼送辛渐}

\poet[唐]王昌龄 \vs

\poetry

寒雨连江夜入吴,

平明送客楚山孤。

洛阳亲友如相问,

一片冰心在玉壶。

\subsection{塞下曲}

\marginnote{\xpinyin{纶}{lun2}}

\poet[唐]卢纶 \vs

\poetry

\marginnote{\xpinyin{单}{chan2} \xpinyin{遁}{dun4}}

月黑雁飞高,单于夜遁逃。

欲将轻骑逐,大雪满弓刀。

\subsection{墨梅}

\marginnote{\xpinyin{冕}{mian3}}

\poet[元]王冕 \vs

\poetry

\marginnote{\xpinyin{砚}{yan4}}

我家洗砚池头树,

朵朵花开淡墨痕。

不要人夸好颜色,

\marginnote{\xpinyin{乾}{qian2} \xpinyin{坤}{kun1}}

只留清气满乾坤。

\newpage

\section{五年级上册}

\subsection{蝉}

\marginnote{\xpinyin{虞}{yu2}}

\poet[唐]虞世南 \vs

\poetry

\marginnote{\xpinyin{緌}{rui2}}

垂緌饮清露,流响出疏桐。

\marginnote{\xpinyin{藉}{jie4}}

居高声自远,非是藉秋风。

\subsection{乞巧}

\poet[唐]林杰 \vs

\poetry

七夕今宵看碧霄,

牵牛织女渡河桥。

家家乞巧望秋月,

穿尽红丝几万条。

\subsection{示儿}

\poet[宋]陆游 \vs

\poetry

死去元知万事空,

但悲不见九州同。

王师北定中原日,

\marginnote{\xpinyin{乃}{nai3}}

家祭无忘告乃翁。

\subsection{题临安邸}

\marginnote{\xpinyin{邸}{di3}}

\poet[宋]林升 \vs

\poetry

山外青山楼外楼,

西湖歌舞几时休?

\marginnote{\xpinyin{熏}{xun1}}

暖风熏得游人醉,

\marginnote{\xpinyin{汴}{bian4}}

直把杭州作汴州。

\subsection{己亥杂诗}

\marginnote{\xpinyin{亥}{hai4} \xpinyin{龚}{gong1}}

\poet[清]龚自珍 \vs

\poetry

\marginnote{\xpinyin{恃}{shi4}}

九州生气恃风雷,

\marginnote{\xpinyin{喑}{yin1}}

万马齐喑究可哀。

\marginnote{\xpinyin{擞}{sou3}}

我劝天公重抖擞,

不拘一格降人才。

\subsection{山居秋暝}

\marginnote{\xpinyin{暝}{ming2}}

\poet[唐]王维 \vs

\poetry

空山新雨后,天气晚来秋。

明月松间照,清泉石上流。

\marginnote{\xpinyin{浣}{huan4}}

竹喧归浣女,莲动下渔舟。

\marginnote{\xpinyin{歇}{xie1}}

随意春芳歇,王孙自可留。

\subsection{枫桥夜泊}

\poet[唐]张继 \vs

\poetry

月落乌啼霜满天,

江枫渔火对愁眠。

姑苏城外寒山寺,

夜半钟声到客船。

\subsection{长相思}

\poet[清]纳兰性德 \vs

\poetry

山一程,

水一程,

\marginnote{\xpinyin{榆}{yu2} \xpinyin{畔}{pan4}}

身向榆关那畔行,

夜深千帐灯。

\marginnote{\xpinyin{更}{geng1}}

风一更,

雪一更,

\marginnote{\xpinyin{聒}{guo1}}

聒碎乡心梦不成,

故园无此声。

\subsection{渔歌子}

\poet[唐]张志和 \vs

\poetry

西塞山前白鹭飞,

\marginnote{\xpinyin{鳜}{gui4}}

桃花流水鳜鱼肥。

\marginnote{\xpinyin{箬}{ruo4}}

青箬笠,绿蓑衣,

斜风细雨不须归。

\subsection{观书有感(其一)}

\marginnote{\xpinyin{熹}{xi1}}

\poet[宋]朱熹 \vs

\poetry

半亩方塘一鉴开,

\marginnote{\xpinyin{徘}{pai2} \xpinyin{徊}{huai2}}

天光云影共徘徊。

问渠那得清如许?

\marginnote{\xpinyin{为}{wei4}}

为有源头活水来。

\subsection{观书有感(其二)}

\poet[宋]朱熹 \vs

\poetry

昨夜江边春水生,

蒙冲巨舰一毛轻。

向来枉费推移力,

此日中流自在行。

\newpage

\section{五年级下册}

\subsection{四时田园杂兴(其三十一)}

\poet[宋]范成大 \vs

\poetry

\marginnote{\xpinyin{昼}{zhou4} \xpinyin{耘}{yun2}}

昼出耘田夜绩麻,

村庄儿女各当家。

\marginnote{\xpinyin{供}{gong4}}

童孙未解供耕织,

也傍桑阴学种瓜。

\subsection{稚子弄冰}

\marginnote{\xpinyin{稚}{zhi4}}

\poet[宋]杨万里 \vs

\poetry

稚子金盆脱晓冰,

\marginnote{\xpinyin{铮}{zheng1}}

彩丝穿取当银铮。

\marginnote{\xpinyin{磬}{qing4}}

敲成玉磬穿林响,

忽作玻璃碎地声。

\subsection{村晚}

\poet[宋]雷震 \vs

\poetry

\marginnote{\xpinyin{陂}{bei1}}

草满池塘水满陂,

\marginnote{\xpinyin{漪}{yi1}}

山衔落日浸寒漪。

牧童归去横牛背,

短笛无腔信口吹。

\subsection{游子吟}

\poet[唐]孟郊 \vs

\poetry

慈母手中线,游子身上衣。

临行密密缝,意恐迟迟归。

谁言寸草心,报得三春晖。

\subsection{鸟鸣涧}

\poet[唐]王维 \vs

\poetry

人闲桂花落,夜静春山空。

月出惊山鸟,时鸣春涧中。

\subsection{从军行}

\poet[唐]王昌龄 \vs

\poetry

青海长云暗雪山,

孤城遥望玉门关。

黄沙百战穿金甲,

\marginnote{\xpinyin{还}{huan2}}

不破楼兰终不还。

\subsection{秋夜将晓出篱门迎凉有感}

\poet[宋]陆游 \vs

\poetry

三万里河东入海,

\marginnote{\xpinyin{仞}{ren4} \xpinyin{岳}{yue4}}

五千仞岳上摩天。

遗民泪尽胡尘里,

南望王师又一年。

\subsection{闻官军收河南河北}

\poet[唐]杜甫 \vs

\poetry

\marginnote{\xpinyin{蓟}{ji4}}

剑外忽传收蓟北,

\marginnote{\xpinyin{涕}{ti4} \xpinyin{裳}{chang2}}

初闻涕泪满衣裳。

却看妻子愁何在,

\marginnote{\xpinyin{卷}{juan3}}

漫卷诗书喜欲狂。

白日放歌须纵酒,

青春作伴好还乡。

即从巴峡穿巫峡,

\marginnote{\xpinyin{襄}{xiang1}}

便下襄阳向洛阳。

\subsection{凉州词}

\poet[唐]王之涣 \vs

\poetry

黄河远上白云间,

一片孤城万仞山。

\marginnote{\xpinyin{羌}{qiang1}}

羌笛何须怨杨柳,

春风不度玉门关。

\subsection{黄鹤楼送孟浩然之广陵}

\poet[唐]李白 \vs

\poetry

故人西辞黄鹤楼,

烟花三月下扬州。

孤帆远影碧空尽,

唯见长江天际流。

\subsection{乡村四月}

\marginnote{\xpinyin{卷}{juan4}}

\poet[宋]翁卷 \vs

\poetry

绿遍山原白满川,

子规声里雨如烟。

乡村四月闲人少,

才了蚕桑又插田。

\newpage

\section{六年级上册}

\subsection{宿建德江}

\poet[唐]孟浩然 \vs

\poetry

\marginnote{\xpinyin{渚}{zhu3}}

移舟泊烟渚,日暮客愁新。

野旷天低树,江清月近人。

\subsection{六月二十七日望湖楼醉书}

\poet[宋]苏轼 \vs

\poetry

黑云翻墨未遮山,

白雨跳珠乱入船。

卷地风来忽吹散,

望湖楼下水如天。

\subsection{西江月·夜行黄沙道中}

\marginnote{\xpinyin{见}{xian4}}

\poet[宋]辛弃疾 \vs

\poetry

明月别枝惊鹊,

清风半夜鸣蝉。

稻花香里说丰年,

听取蛙声一片。

七八个星天外,

两三点雨山前。

旧时茅店社林边,

路转溪桥忽见。

\subsection{过故人庄}

\poet[唐]孟浩然 \vs

\poetry

\marginnote{\xpinyin{黍}{shu3}}

故人具鸡黍,

邀我至田家。

绿树村边合,

青山郭外斜。

\marginnote{\xpinyin{圃}{pu3}}

开轩面场圃,

把酒话桑麻。

待到重阳日,

还来就菊花。

\subsection{七律·长征}

\poet 毛泽东 \vs

\poetry

红军不怕远征难,

万水千山只等闲。

\marginnote{\xpinyin{逶}{wei1} \xpinyin{迤}{yi2}}

五岭逶迤腾细浪,

\marginnote{\xpinyin{磅}{pang2} \xpinyin{礴}{bo2}}

乌蒙磅礴走泥丸。

\marginnote{\xpinyin{丸}{wan2}}

金沙水拍云崖暖,

大渡桥横铁索寒。

\marginnote{\xpinyin{岷}{min2}}

更喜岷山千里雪,

三军过后尽开颜。

\subsection{春日}

\poet[宋]朱熹 \vs

\poetry

\marginnote{\xpinyin{泗}{si4} \xpinyin{滨}{bin1}}

胜日寻芳泗水滨,

无边光景一时新。

等闲识得东风面,

万紫千红总是春。

\subsection{回乡偶书}

\poet[唐]贺知章 \vs

\poetry

少小离家老大回,

\marginnote{\xpinyin{鬓}{bin4}}

乡音无改鬓毛衰。

儿童相见不相识,

笑问客从何处来。

\subsection{浪淘沙(其一)}

\poet[唐]刘禹锡 \vs

\poetry

九曲黄河万里沙,

\marginnote{\xpinyin{簸}{bo3}}

浪淘风簸自天涯。

如今直上银河去,

同到牵牛织女家。

\subsection{江南春}

\poet[唐]杜牧 \vs

\poetry

千里莺啼绿映红,

\marginnote{\xpinyin{郭}{guo1}}

水村山郭酒旗风。

南朝四百八十寺,

多少楼台烟雨中。

\subsection{书湖阴先生壁}

\poet[宋]王安石 \vs

\poetry

\marginnote{\xpinyin{檐}{yan2}}

茅檐长扫净无苔,

\marginnote{\xpinyin{畦}{qi2}}

花木成畦手自栽。

一水护田将绿绕,

\marginnote{\xpinyin{闼}{ta4}}

两山排闼送青来。

\newpage

\section{六年级下册}

\subsection{寒食}

\marginnote{\xpinyin{翃}{hong2}}

\poet[唐]韩翃 \vs

\poetry

春城无处不飞花,

\marginnote{\xpinyin{御}{yu4}}

寒食东风御柳斜。

日暮汉宫传蜡烛,

轻烟散入五侯家。

\subsection{迢迢牵牛星}

\marginnote{\xpinyin{迢}{tiao2} \xpinyin{皎}{jiao3} }

\poet《古诗十九首》 \vs

\poetry

\marginnote{\xpinyin{纤}{xian1} \xpinyin{擢}{zhuo2}}

迢迢牵牛星,皎皎河汉女。

\marginnote{\xpinyin{札}{zha2} \xpinyin{杼}{zhu4}}

纤纤擢素手,札札弄机杼。

\marginnote{\xpinyin{涕}{ti4}}

终日不成章,泣涕零如雨。

河汉清且浅,相去复几许。

\marginnote{\xpinyin{脉}{mo4}}

盈盈一水间,脉脉不得语。

\subsection{十五夜望月}

\poet[唐]王建 \vs

\poetry

中庭地白树栖鸦,

冷露无声湿桂花。

今夜月明人尽望,

不知秋思落谁家?

\subsection{长歌行}

\poet 汉乐府 \vs

\poetry

\marginnote{\xpinyin{晞}{xi1}}

青青园中葵,朝露待日晞。

阳春布德泽,万物生光辉。

\marginnote{\xpinyin{焜}{kun1}}

常恐秋节至,焜黄华叶衰。

百川东到海,何时复西归?

少壮不努力,老大徒伤悲!

\subsection{马诗}

\poet[唐]李贺 \vs

\poetry

大漠沙如雪,

燕山月似钩。

何当金络脑,

快走踏清秋。

\subsection{石灰吟}

\poet[明]于谦 \vs

\poetry

千锤万凿出深山,

烈火焚烧若等闲。

粉身碎骨浑不怕,

要留清白在人间。

\subsection{竹石}

\marginnote{\xpinyin{燮}{xie4}}

\poet[清]郑燮 \vs

\poetry

咬定青山不放松,

立根原在破岩中。

\marginnote{\xpinyin{劲}{jing4}}

千磨万击还坚劲,

任尔东西南北风。

\subsection{采薇(节选)}

\poet《诗经·小雅》 \vs

\poetry

昔我往矣,杨柳依依。

\marginnote{\xpinyin{霏}{fei1}}

今我来思,雨雪霏霏。

行道迟迟,载渴载饥。

我心伤悲,莫知我哀!

\subsection{送元二使安西}

\poet[唐]王维 \vs

\poetry

\marginnote{\xpinyin{渭}{wei4} \xpinyin{浥}{yi4}}

渭城朝雨浥轻尘,

\marginnote{\xpinyin{舍}{she4}}

客舍青青柳色新。

劝君更尽一杯酒,

西出阳关无故人。

\subsection{春夜喜雨}

\poet[唐]杜甫 \vs

\poetry

好雨知时节,当春乃发生。

随风潜入夜,润物细无声。

野径云俱黑,江船火独明。

\marginnote{\xpinyin{重}{zhong4}}

晓看红湿处,花重锦官城。

\subsection{早春呈水部张十八员外}

\poet[唐]韩愈 \vs

\poetry

\marginnote{\xpinyin{酥}{su1}}

天街小雨润如酥,

草色遥看近却无。

最是一年春好处,

绝胜烟柳满皇都。

\subsection{江上渔者}

\poet[宋]范仲淹 \vs

\poetry

江上往来人,

\marginnote{\xpinyin{鲈}{lu2}}

但爱鲈鱼美。

君看一叶舟,

\marginnote{\xpinyin{没}{mo4}}

出没风波里。

\subsection{泊船瓜洲}

\poet[宋]王安石 \vs

\poetry

京口瓜洲一水间,

钟山只隔数重山。

春风又绿江南岸,

明月何时照我还。

\subsection{游园不值}

\poet[宋]叶绍翁 \vs

\poetry

\marginnote{\xpinyin{应}{ying1} \xpinyin{屐}{ji1}}

应怜屐齿印苍苔,

\marginnote{\xpinyin{扉}{fei1}}

小扣柴扉久不开。

春色满园关不住,

一枝红杏出墙来。

\subsection{卜算子·送鲍浩然之浙东}

\marginnote{\xpinyin{鲍}{bao4}}

\poet[宋]王观 \vs

\poetry

水是眼波横,

山是眉峰聚。

\marginnote{那,同\xpinyin{哪}{na3}}

欲问行人去那边?

眉眼盈盈处。

才始送春归,

又送君归去。

若到江南赶上春,

千万和春住。

\subsection{浣溪沙}

\marginnote{\xpinyin{浣}{huan4} \xpinyin{蕲}{qi2}}

\poet[宋]苏轼 \vs \vs

游蕲水清泉寺,

寺临兰溪,溪水西流。 \vs

\poetry

山下兰芽短浸溪,

松间沙路净无泥,

\marginnote{\xpinyin{潇}{xiao1}}

潇潇暮雨子规啼。

谁道人生无再少?

门前流水尚能西!

休将白发唱黄鸡。

\subsection{清平乐}

\marginnote{\xpinyin{乐}{yue4}}

\poet[宋]黄庭坚 \vs

\poetry

春归何处?

寂寞无行路。

若有人知春去处,

唤取归来同住。

春无踪迹谁知?

 除非问取黄鹂。

\marginnote{\xpinyin{啭}{zhuan4}}

百啭无人能解,

因风飞过蔷薇。

\newpage

\section{增补篇目}

\subsection{秋浦歌}

\poet[唐]李白 \vs

\poetry

白发三千丈,缘愁似个长。

不知明镜里,何处得秋霜。

\subsection{杂诗}

\poet[唐]王维 \vs

\poetry

君自故乡来,

应知故乡事。

\marginnote{\xpinyin{绮}{qi3}}

来日绮窗前,

\marginnote{\xpinyin{著}{zhuo2}}

寒梅著花未?

\subsection{江畔独步寻花}

\poet[唐]杜甫 \vs

\poetry

\marginnote{\xpinyin{蹊}{xi1}}

黄四娘家花满蹊,

千朵万朵压枝低。

留连戏蝶时时舞,

自在娇莺恰恰啼。

\subsection{乐游原}

\poet[唐]李商隐 \vs

\poetry

向晚意不适,

驱车登古原。

夕阳无限好,

只是近黄昏。

\subsection{七步诗}

\marginnote{\xpinyin{羹}{geng1}}

\poet[三国]曹植 \vs

\poetry

\marginnote{\xpinyin{漉}{lu4} \xpinyin{菽}{shu1}}

煮豆持作羹,漉菽以为汁。

\marginnote{\xpinyin{萁}{qi2} \xpinyin{釜}{fu3}}

萁在釜下燃,豆在釜中泣。

本自同根生,相煎何太急?

\subsection{寒菊}

\poet[宋]郑思肖 \vs

\poetry

花开不并百花丛,

独立疏篱趣未穷。

宁可枝头抱香死,

何曾吹落北风中。

\subsection{赠花卿}

\poet[唐]杜甫 \vs

\poetry

\mbox{锦城丝管日纷纷,半入江风半入云。}

\mbox{此曲只应天上有,人间能得几回闻。}

\subsection{逢学宿芙蓉山主人}

\poet[唐]刘长卿 \vs

\poetry

日暮苍山远,天寒白屋贫。

\marginnote{\xpinyin{吠}{fei4}}

柴门闻犬吠,风雪夜归人。

\subsection{竹枝词}

\poet[唐]刘禹锡 \vs

\poetry

杨柳青青江水平,

闻郎江上唱歌声。

东边日出西边雨,

道是无晴却有晴。

\subsection{相思}

\poet[唐]王维 \vs

\poetry

红豆生南国,春来发几枝。

\marginnote{\xpinyin{撷}{xie2}}

愿君多采撷,此物最相思。

\subsection{蚕妇}

\marginnote{\xpinyin{俞}{yu2}}

\poet[宋]张俞 \vs

\poetry

昨日入城市,归来泪满巾。

\marginnote{\xpinyin{绮}{qi3}}

遍身罗绮者,不是养蚕人。

\subsection{冬夜读书示子聿}

\marginnote{\xpinyin{聿}{yu4}}

\poet[宋]陆游 \vs

\poetry

古人学问无遗力,

少壮工夫老始成。

纸上得来终觉浅,

绝知此事要躬行。

\subsection{乌衣巷}

\poet[唐]刘禹锡 \vs

\poetry

朱雀桥边野草花,

乌衣巷口夕阳斜。

旧时王谢堂前燕,

飞入寻常百姓家。

\subsection{菩萨蛮·书江西造口壁}

\marginnote{\xpinyin{蛮}{man2}}

\poet[宋]辛弃疾 \vs

\poetry

郁孤台下清江水,

中间多少行人泪?

西北望长安,可怜无数山。

青山遮不住,毕竟东流去。

\marginnote{\xpinyin{鹧}{zhe4} \xpinyin{鸪}{gu1}}

江晚正愁余,山深闻鹧鸪。

\subsection{题乌江亭}

\poet[唐]杜牧 \vs

\poetry

胜败兵家事不期,

包羞忍耻是男儿。

江东子弟多才俊,

卷土重来未可知。

\subsection{华山}

\marginnote{\xpinyin{华}{hua4} \xpinyin{寇}{kou4}}

\poet[宋]寇准 \vs

\poetry

只有天在上,更无山与齐。

举头红日近,回首白云低。

\subsection{绝句}

\poet[唐]杜甫 \vs

\poetry

江碧鸟逾白,山青花欲燃。

今春看又过,何日是归年?

\subsection{咏风}

\marginnote{\xpinyin{虞}{yu2}}

\poet[唐]虞世南 \vs

\poetry

逐舞飘轻袖,

传歌共绕梁。

动枝生乱影,

吹花送远香。

\subsection{渭水思秦川}

\marginnote{\xpinyin{岑}{cen2} \xpinyin{参}{shen1}}

\poet[唐]岑参 \vs

\poetry

\marginnote{\xpinyin{雍}{yong1}}

渭水东流去,何时到雍州。

凭添两行泪,寄向故园流。

\subsection{早梅}

\poet[明]道源 \vs

\poetry

万树寒无色,南枝独有花。

香闻流水处,影落野人家。

\subsection{陶者}

\poet[宋]梅尧臣 \vs

\poetry

\marginnote{\xpinyin{尧}{yao2}}

陶尽门前土,屋上无片瓦。

\marginnote{\xpinyin{沾}{zhan1} \xpinyin{鳞}{lin2}}

十指不沾泥,鳞鳞居大厦。

\subsection{春雪}

\poet[唐]韩愈 \vs

\poetry

新年都未有芳华,

二月初惊见草芽。

白雪却嫌春色晚,

故穿庭树作飞花。

\subsection{长干行(节选)}

\poet[唐]李白 \vs

\poetry

郎骑竹马来,绕床弄青梅。

同居长干里,两小无嫌猜。

\subsection{长干行}

\marginnote{\xpinyin{颢}{hao4}}

\poet[唐]崔颢 \vs

\poetry

\marginnote{\xpinyin{妾}{qie4}}

君家何处住? 妾住在横塘。

停船暂借问,或恐是同乡。

\subsection{牧童}

\poet[唐]吕岩 \vs

\poetry

草铺横野六七里,

笛弄晚风三四声。

归来饱饭黄昏后,

不脱蓑衣卧月明。

\subsection{秋思}

\marginnote{\xpinyin{籍}{ji2}}

\poet[唐]张籍 \vs

\poetry

洛阳城里见秋风,

\marginnote{\xpinyin{重}{chong2}}

欲作家书意万重。

复恐匆匆说不尽,

行人临发又开封。

\subsection{官仓鼠}

\marginnote{\xpinyin{邺}{ye4}}

\poet[唐]曹邺 \vs

\poetry

官仓老鼠大如斗,

见人开仓亦不走。

健儿无粮百姓饥,

谁遣朝朝入君口。

\subsection{画眉鸟}

\poet[宋]欧阳修 \vs

\poetry

\marginnote{\xpinyin{啭}{zhuan4}}

百啭千声随意移,

山花红紫树高低。

始知锁向金笼听,

不及林间自在啼。

\subsection{丰乐亭游春}

\poet[宋]欧阳修 \vs

\poetry

红树青山日欲斜,

长郊草色绿无涯。

游人不管春将老,

来往亭前踏落花。

\subsection{赤日炎炎似火烧}

\poet[宋朝民歌] \vs

\poetry

\mbox{赤日炎炎似火烧,野田禾稻半枯焦。}

\mbox{农夫心内如汤煮,公子王孙把扇摇。}

\subsection{终南望余雪}

\poet[唐]祖咏 \vs

\poetry

终南阴岭秀,积雪浮云端。

\marginnote{\xpinyin{霁}{ji4}}

林表明霁色,城中增暮寒。

\subsection{江上}

\poet[宋]王安石 \vs

\poetry

江北秋阴一半开,

晚云含雨却低回。

\marginnote{\xpinyin{缭}{liao2} \xpinyin{绕}{rao4}}

青山缭绕疑无路,

忽见千帆隐映来。

\subsection{梅花绝句(其一)}

\poet[宋]陆游 \vs

\poetry

\marginnote{\xpinyin{坼}{che4}}

闻道梅花坼晓风,

雪堆遍满四山中。

何方可化身千亿,

一树梅前一放翁。

\subsection{安宁道中即事}

\poet[清]王文治 \vs

\poetry

夜来春雨润垂杨,

春水新生不满塘。

日暮平原风过处,

菜花香杂豆花香。

\subsection{初秋行圃}

\marginnote{\xpinyin{圃}{pu3}}

\poet[宋]杨万里 \vs

\poetry

落日无情最有情,

遍催万树暮蝉鸣。

\marginnote{\xpinyin{咫}{zhi3}}

听来咫尺无寻处,

寻到旁边却不声。

\subsection{明日歌}

\poet[明]钱福 \vs

\poetry

明日复明日,明日何其多。

\marginnote{\xpinyin{蹉}{cuo1} \xpinyin{跎}{tuo2}}

我生待明日,万事成蹉跎。

世人若被明日累,

春去秋来老将至。

朝看水东流,暮看日西坠。

百年明日能几何?

请君听我明日歌。

\subsection{海棠}

\poet[宋]苏轼 \vs

\poetry

\marginnote{\xpinyin{袅}{niao3}}

东风袅袅泛崇光,

香雾空蒙月转廊。

只恐夜深花睡去,

故烧高烛照红妆。

\subsection{早梅}

\poet[唐]张谓 \vs

\poetry

一树寒梅白玉条,

\marginnote{\xpinyin{迥}{jiong3}}

迥临村路傍溪桥。

不知近水花先发,

疑是经冬雪未销。

\subsection{野步}

\poet[宋]周密 \vs

\poetry

\marginnote{\xpinyin{陇}{long3}}

麦陇风来翠浪斜,

草根肥水噪新蛙。

羡他无事双蝴蝶,

烂醉东风野草花。

\subsection{塞下曲(其二)}

\poet[唐]卢纶 \vs

\poetry

林暗草惊风,

将军夜引弓。

平明寻白羽,

\marginnote{\xpinyin{没}{mo4}}

没在石棱中。

\subsection{送友人}

\poet[唐]李白 \vs

\poetry

青山横北郭,白水绕东城。

\marginnote{\xpinyin{蓬}{peng2}}

此地一为别,孤蓬万里征。

浮云游子意,落日故人情。

\marginnote{\xpinyin{兹}{zi1}}

挥手自兹去,萧萧班马鸣。

\subsection{题李凝幽居}

\marginnote{\xpinyin{凝}{ning2} \xpinyin{幽}{you1}}

\poet[唐]贾岛 \vs

\poetry

闲居少邻并,草径入荒园。

鸟宿池边树,僧敲月下门。

过桥分野色,移石动云根。

暂去还来此,幽期不负言。

\subsection{田园乐(其六)}

\poet[唐]王维 \vs

\poetry

桃红复含宿雨,

柳绿更带朝烟。

花落家童未扫,

莺啼山客犹眠。

\subsection{遗爱寺}

\poet[唐]白居易 \vs

\poetry

弄石临溪坐,寻花绕寺行。

时时闻鸟语,处处是泉声。

\subsection{关山月}

\poet[唐]李白 \vs

\poetry

明月出天山,苍茫云海间。

长风几万里,吹度玉门关。

\marginnote{\xpinyin{窥}{kui1}}

汉下白登道,胡窥青海湾。

由来征战地,不见有人还。

\marginnote{\xpinyin{戍}{shu4}}

戍客望边色,思归多苦颜。

高楼当此夜,叹息未应闲。

\subsection{鸟}

\poet[唐]白居易 \vs

\poetry

谁道群生性命微,

一般骨肉一般皮。

劝君莫打枝头鸟,

\marginnote{\xpinyin{巢}{chao2}}

子在巢中望母归。

\subsection{花影}

\poet[宋]苏轼 \vs

\poetry

重重叠叠上瑶台,

几度呼童扫不开。

刚被太阳收拾去,

却教明月送将来。

\subsection{清平调(其一)}

\poet[唐]李白 \vs

\poetry

\marginnote{\xpinyin{裳}{chang2}}

云想衣裳花想容,

\marginnote{\xpinyin{槛}{jian4}}

春风拂槛露华浓。

若非群玉山头见,

会向瑶台月下逢。

\subsection{山中送别}

\poet[唐]王维 \vs

\poetry

\marginnote{\xpinyin{罢}{ba4}}

山中相送罢,

日暮掩柴扉。

春草明年绿,

王孙归不归。

\subsection{绝句}

\poet[宋]志南 \vs

\poetry

古木阴中系短篷,

\marginnote{\xpinyin{藜}{li2}}

杖藜扶我过桥东。

沾衣欲湿杏花雨,

吹面不寒杨柳风。

\subsection{子夜吴歌(其三)}

\poet[唐]李白 \vs

\poetry

\marginnote{\xpinyin{捣}{dao3}}

长安一片月,万户捣衣声。

秋风吹不尽,总是玉关情。

\marginnote{\xpinyin{虏}{lu3}}

何日平胡虏,良人罢远征?

\subsection{夜泊牛渚怀古}

\marginnote{\xpinyin{渚}{zhu3}}

\poet[唐]李白 \vs

\poetry

牛渚西江夜,青天无片云。

登舟望秋月,空忆谢将军。

余亦能高咏,斯人不可闻。

明朝挂帆席,枫叶落纷纷。

\subsection{听蜀僧濬弹琴}

\marginnote{\xpinyin{濬}{jun4}}

\poet[唐]李白 \vs

\poetry

蜀僧抱绿绮,西下峨眉峰。

\marginnote{\xpinyin{壑}{he4}}

为我一挥手,如听万壑松。

\marginnote{\xpinyin{馀}{yu2}}

客心洗流水,馀响入霜钟。

不觉碧山暮,秋云暗几重。

\subsection{送上人}

\poet[唐]刘长卿 \vs

\poetry

孤云将野鹤,

岂向人间住。

莫买沃洲山,

时人已知处。

\subsection{题金陵渡}

\marginnote{\xpinyin{祜}{hu4}}

\poet[唐]张祜 \vs

\poetry

金陵津渡小山楼,

一宿行人自可愁。

潮落夜江斜月里,

两三星火是瓜洲。

\subsection{月下独酌(其一)}

\marginnote{\xpinyin{酌}{zhuo2}}

\poet[唐]李白 \vs

\poetry

花间一壶酒,独酌无相亲。

\marginnote{\xpinyin{邀}{yao1}}

举杯邀明月,对影成三人。

月既不解饮,影徒随我身。

暂伴月将影,行乐须及春。

\marginnote{\xpinyin{徘}{pai2} \xpinyin{徊}{huai2}}

我歌月徘徊,我舞影零乱。

醒时同交欢,醉后各分散。

\marginnote{\xpinyin{邈}{miao3}}

永结无情游,相期邈云汉。

\end{document}
